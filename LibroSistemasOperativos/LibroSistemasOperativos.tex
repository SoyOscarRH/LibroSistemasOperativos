% ****************************************************************************************
% *********************        SISTEMAS OPERATIVOS            ****************************
% ****************************************************************************************

% =======================================================
% =======         HEADER FOR DOCUMENT        ============
% =======================================================
    % *********   DOCUMENT ITSELF   **************
    \documentclass[12pt, fleqn]{report}                             %Type of docuemtn and size of font and left eq
    \usepackage[margin=1.2in]{geometry}                             %Margins and Geometry pacakge
    \usepackage{ifthen}                                             %Allow simple programming
    \usepackage{hyperref}                                           %Create MetaData for a PDF and LINKS!
    \setlength{\parindent}{0pt}                                     %Eliminate ugly indentation
    \author{Oscar Andrés Rosas}                                     %Who I am

    % *********   LANGUAJE AND UFT-8   *********
    \usepackage[spanish]{babel}                                     %Please use spanish
    \usepackage[utf8]{inputenc}                                     %Please use spanish - UFT
    \usepackage[T1]{fontenc}                                        %Please use spanish
    \usepackage{textcmds}                                           %Allow us to use quoutes
    \usepackage{changepage}                                         %Allow us to use identate paragraphs

    % *********   MATH AND HIS STYLE  *********
    \usepackage{ntheorem, amsmath, amssymb, amsfonts}               %All fucking math, I want all!
    \usepackage{mathrsfs, mathtools, empheq}                        %All fucking math, I want all!
    \usepackage{centernot}                                          %Allow me to negate a symbol
    \decimalpoint                                                   %Use decimal point

    % *********   GRAPHICS AND IMAGES *********
    \usepackage{graphicx}                                           %Allow to create graphics
    \usepackage{wrapfig}                                            %Allow to create images
    \graphicspath{ {Graphics/} }                                    %Where are the images :D

    % *********   LISTS AND TABLES ***********
    \usepackage{listings}                                           %We will be using code here
    \usepackage[inline]{enumitem}                                   %We will need to enumarate
    \usepackage{tasks}                                              %Horizontal lists
    \usepackage{longtable}                                          %Lets make tables awesome
    \usepackage{booktabs}                                           %Lets make tables awesome
    \usepackage{tabularx}                                           %Lets make tables awesome
    \usepackage{multirow}                                           %Lets make tables awesome
    \usepackage{multicol}                                           %Create multicolumns

    % *********   HEADERS AND FOOTERS ********
    \usepackage{fancyhdr}                                           %Lets make awesome headers/footers
    \pagestyle{fancy}                                               %Lets make awesome headers/footers
    \setlength{\headheight}{16pt}                                   %Top line
    \setlength{\parskip}{0.5em}                                     %Top line
    \renewcommand{\footrulewidth}{0.5pt}                            %Bottom line

    \lhead{                                                         %Left Header
        \hyperlink{chapter.\arabic{chapter}}                        %Make a link to the current chapter
        {\normalsize{\textsc{\nouppercase{\leftmark}}}}             %And fot it put the name
    }

    \rhead{                                                         %Right Header
        \hyperlink{section.\arabic{chapter}.\arabic{section}}       %Make a link to the current chapter
            {\footnotesize{\textsc{\nouppercase{\rightmark}}}}      %And fot it put the name
    }

    \rfoot{\textsc{\small{\hyperref[sec:Index]{Ve al Índice}}}}     %This will always be a footer  

    \fancyfoot[L]{                                                  %Algoritm for a changing footer
        \ifthenelse{\isodd{\value{page}}}                           %IF ODD PAGE:
            {\href{https://compilandoconocimiento.com/yo/}          %DO THIS:
                {\footnotesize                                      %Send the page
                    {\textsc{Oscar Andrés Rosas}}}}                 %Send the page
            {\href{https://compilandoconocimiento.com}              %ELSE DO THIS: 
                {\footnotesize                                      %Send the author
                    {\textsc{Compilando Conocimiento}}}}            %Send the author
    }
    
    
    
% ========================================
% ===========   COMMANDS    ==============
% ========================================

    % =====  GENERAL TEXT  ==========
    \newcommand \Quote {\qq}                                        %Use: \Quote to use quotes
    \newcommand \Over {\overline}                                   %Use: \Bar to use just for short
    \newcommand \ForceNewLine {$\Space$\\}                          %Use it in theorems for example
    
    \newenvironment{Indentation}[1][0.75em]                         %Use: \begin{Inde...}[Num]...\end{Inde...}
    {\begin{adjustwidth}{#1}{}}                                     %If you dont put nothing i will use 0.75 em
    {\end{adjustwidth}}                                             %This indentate a paragraph
    \newenvironment{SmallIndentation}[1][0.75em]                    %Use: The same that we upper one, just 
    {\begin{adjustwidth}{#1}{}\begin{footnotesize}}                 %footnotesize size of letter by default
    {\end{footnotesize}\end{adjustwidth}}                           %that's it
        
    % =====  GENERAL MATH  ==========
    \DeclareMathOperator \Space {\quad}                             %Use: \Space for a cool mega space
    \DeclareMathOperator \MiniSpace {\;}                            %Use: \Space for a cool mini space
    \newcommand \Such {\MiniSpace|\MiniSpace}                       %Use: \Such like in sets
    \newcommand \Also {\Space \text{y} \Space}                      %Use: \Also so it's look cool
    \newcommand \Remember[1]{\Space\text{\scriptsize{#1}}}          %Use: \Remember so it's look cool

    \newtheorem{Theorem}{Teorema}[section]                          %Use: \begin{Theorem}[Name]\label{Nombre}...
    \newtheorem{Corollary}{Colorario}[Theorem]                      %Use: \begin{Corollary}[Name]\label{Nombre}...
    \newtheorem{Lemma}[Theorem]{Lemma}                              %Use: \begin{Lemma}[Name]\label{Nombre}...
    \newtheorem{Definition}{Definición}[section]                    %Use: \begin{Definition}[Name]\label{Nombre}...

    \newcommand{\Set}[1]{\left\{ \MiniSpace #1 \MiniSpace \right\}} %Use: \Set {Info}
    \newcommand{\Brackets}[1]{\left[ #1 \right]}                    %Use: \Brackets {Info} 
    \newcommand{\Wrap}[1]{\left( #1 \right)}                        %Use: \Wrap {Info} 
    \newcommand{\pfrac}[2]{\Wrap{\dfrac{#1}{#2}}}                   %Use: Put fractions in parentesis

    \newenvironment{MultiLineEquation}[1]                           %Use: To create MultiLine equations
        {\begin{equation}\begin{alignedat}{#1}}                     %Use: \begin{Multi..}{Num. de Columnas}
        {\end{alignedat}\end{equation}}                             %And.. that's it!
    \newenvironment{MultiLineEquation*}[1]                          %Use: To create MultiLine equations
        {\begin{equation*}\begin{alignedat}{#1}}                    %Use: \begin{Multi..}{Num. de Columnas}
        {\end{alignedat}\end{equation*}}                            %And.. that's it!


    % =====  LOGIC  ==================
    \DeclareMathOperator \doublearrow {\leftrightarrow}             %Use: \doublearrow for a double arrow
    \newcommand \lequal {\MiniSpace \Leftrightarrow \MiniSpace}     %Use: \lequal for a double arrow
    \newcommand \linfire {\MiniSpace \Rightarrow \MiniSpace}        %Use: \lequal for a double arrow
    \newcommand \longto {\longrightarrow}                           %Use: \longto for a long arrow

    % =====  NUMBER THEORY  ==========
    \DeclareMathOperator \Naturals  {\mathbb{N}}                     %Use: \Naturals por Notation
    \DeclareMathOperator \Primes    {\mathbb{P}}                     %Use: \Naturals por Notation
    \DeclareMathOperator \Integers  {\mathbb{Z}}                     %Use: \Integers por Notation
    \DeclareMathOperator \Racionals {\mathbb{Q}}                     %Use: \Racionals por Notation
    \DeclareMathOperator \Reals     {\mathbb{R}}                     %Use: \Reals por Notation
    \DeclareMathOperator \Complexs  {\mathbb{C}}                     %Use: \Complex por Notation

    % === LINEAL ALGEBRA & VECTORS ===
    \DeclareMathOperator \LinealTransformation {\mathcal{T}}        %Use: \LinealTransformation for a cool T

    \newcommand{\pVector}[1]{                                       %Use: \pVector {Matrix Notation} use parentesis
        \ensuremath{\begin{pmatrix}#1\end{pmatrix}}                 %Example: \pVector{a\\b\\c} or \pVector{a&b&c} 
    }
    \newcommand{\lVector}[1]{                                       %Use: \lVector {Matrix Notation} use a abs 
        \ensuremath{\begin{vmatrix}#1\end{vmatrix}}                 %Example: \lVector{a\\b\\c} or \lVector{a&b&c} 
    }
    \newcommand{\bVector}[1]{                                       %Use: \bVector {Matrix Notation} use a brackets 
        \ensuremath{\begin{bmatrix}#1\end{bmatrix}}                 %Example: \bVector{a\\b\\c} or \bVector{a&b&c} 
    }
    \newcommand{\Vector}[1]{                                        %Use: \Vector {Matrix Notation} no parentesis
        \ensuremath{\begin{matrix}#1\end{matrix}}                   %Example: \Vector{a\\b\\c} or \Vector{a&b&c}
    }
    \newcommand{\uvec}[1]{\boldsymbol{\hat{\textbf{$#1$}}}}         %Use: Unitary Vector

    % MATRIX
    \makeatletter                                                   %Example: \begin{matrix}[cc|c]
    \renewcommand*\env@matrix[1][*\c@MaxMatrixCols c] {             %WTF! IS THIS
        \hskip -\arraycolsep                                        %WTF! IS THIS
        \let\@ifnextchar\new@ifnextchar                             %WTF! IS THIS
        \array{#1}                                                  %WTF! IS THIS
    }                                                               %WTF! IS THIS
    \makeatother                                                    %WTF! IS THIS

    % TRIGONOMETRIC FUNCTIONS
    \newcommand{\Cos}[1]{\cos\Wrap{#1}}
    \newcommand{\Sin}[1]{\sin\Wrap{#1}}

    % === CALCULUS ===                               
    \newcommand \Derivate[2] {\dfrac{d #1}{d#2}}                    %Use: Derivate Notation
    \newcommand \Partial[2] {\dfrac{\partial #1}{\partial#2}}       %Use: Derivate Partial Notation
    
    \newcommand \pDerivate[2]{\Derivate{\Wrap{#1}}{#2}}             %But with cool parentesis
    \newcommand \pPartial[2]{\Partial{\Wrap{#1}}{#2}}               %Use: Derivate Partial
    
    \newcommand \SemiDerivate[1]{\Wrap{\dfrac{d}{d#1}}}             %Use: Derivate Notation
    \newcommand \SemiPartial[1]{\Wrap{\dfrac{\partial}{\partial#1}}}%Use: Derivate Partial Notation


    % === COMPLEX ANALYSIS ===
    \newcommand \Cis[1]  {\Cos{#1} + i \Sin{#1}}                    %Use: \Cis for cos(x) + i sin(x)
    \newcommand \pCis[1] {\Wrap{\Cis{#1}}}                          %Use: \pCis for the same ut parantesis



% =====================================================
% ============     	  COVER PAGE	   ================
% =====================================================
\begin{document}
\begin{titlepage}

	\center
	% ============ UNIVERSITY NAME AND DATA =========
	\textbf{\textsc{\Large Proyecto Compilando Conocimiento}}\\[1.0cm] 
	\textsc{\Large Programación}\\[1.0cm] 

	% ============ NAME OF THE DOCUMENT  ============
	\rule{\linewidth}{0.5mm} \\[1.0cm]
		{ \huge \bfseries Bases de Sistemas Operativos}\\[1.0cm] 
	\rule{\linewidth}{0.5mm} \\[2.0cm]
	
	% ====== SEMI TITLE ==========
	{\LARGE Una Pequeña (Gran) Introducción}\\[7cm] 
	
	% ============  MY INFORMATION  =================
	\begin{center} \large
    \textbf{\textsc{Autores:}}\\
        Rosas Hernandez Oscar Andrés \\
        Lopez Manriquez Angel
    \end{center}

	\vfill

\end{titlepage}

% =====================================================
% ========                INDICE              =========
% =====================================================
\tableofcontents{}
\label{sec:Index}

\clearpage




% //////////////////////////////////////////////////////////////////////////////////////////////////////////
% ///////////////////////////////////         SISTEMAS DE COORDENADAS      /////////////////////////////////
% //////////////////////////////////////////////////////////////////////////////////////////////////////////
\part{Parte Abstracta}
\clearpage


    % ===============================================================================
    % ===================           DEFINICIONES               ======================
    % ===============================================================================
    \chapter{Introducción}

        % ==============================================
        % ========    REPOSITORIO DE DATOS     =========
        % ==============================================
        \clearpage
        \section{¿Qué es un Sistema Operativo?}
            
            Los sistemas operativos surgen como una solución a la problematica de la administración de un equipo
            de computo, de forma tal que fuese simplificada.
    
            "Un sistema operativo es un programa encargado de controlar todos lor recursos de una copmutadora"

            \subsection{Definición Formal}

                \Quote{Un sistema operativo es un software de base compuesto por un conjunto de administradores
                encargados de la administracion de cada uno de los recursos de un equipo de coputo de forma
                rápida y eficiente.}


                \subsection*{Características}

                    \begin{itemize}
                        \item 
                            Un sistema operativo es un software de base debido a que es una plataforma que permite
                            la creación y ejecución de aplicaciones desarrolladas para el propio sistema.
                            Como software de base, el sistem operativo ofrece interfaces para la creación o ejecución
                            de las aplicaciones desarrolladas.

                        \item
                            Un sistema operativo esta compuesto de un conjunto de administradores, los cuales controlan
                            todos los recursos del equipo de computo, estos administradores son: 

                            \begin{itemize}
                                \item Administrador de Procesos
                                \item Admin. de Memoria 
                                \item Admin. de Entrada / Salida 
                                \item Admin. de Archivos 
                                \item Admin. de Red 
                            \end{itemize}

                        \item
                            Un sistema operativo debe ejecutarse lo más rapido posible, evitando quitarle tiempo de
                            procesamiento a las aplicaciones de los usuarios, por otro lado debe administrar cada uno
                            de los recursos del equipo de cómputo de forma eficiente, maximizando el uso de cada
                            recurso controlado. 
                            La rápidez y eficiencia es uno de los principales u objetivos que un Sistema Operativo debe
                            cumplir durante su ejecución.
                
                    \end{itemize}
                

        % ==============================================
        % ========      TERMINOS COMUNES       =========
        % ==============================================
        \clearpage
        \section{Terminos Básicos}

            \begin{itemize}
                \item
                    \textbf{Spooling: }

                    Técnica que nos permite disminuir el tiempo en el que la CPU no se encuentra
                    realizando ningún trabajo, esto se logra utilizando el disco como buffer de
                    almacenamiento de trabajos.

                \item
                    \textbf{Planificación de Trabajo: }

                    Es la técnica que se encarga de seleccionar cual será e siguiente trabajo
                    ejecutado en la CPU.

                \item
                    \textbf{Multiprogramación: }

                    Técnica utilizada para almacenar múltiples trabajos simultáneamente en la 
                    memoria física (RAM).

                \item
                    \textbf{Tiempo Compartido: }

                    Técnica utilizada para asignar un tiempo de ejecución a cada proceso
                    lo suficientemente corto para conmutar entre ellos.

                \item
                    \textbf{Concurrencia: }

                    Técnica utilizada ejecutar múltiples trabajos bajo la apariencia de
                    simultaneidad o paralelismo mediante una ejecución secuencial.

                \item
                    \textbf{Memoria Virtual: }

                    Técnica utilizada para aumentar o extender la memoria física (RAM) mediante
                    el uso de una pequeña región de disco.

                \item
                    \textbf{Sistema de Archivos: }

                    Estructura de almacenamiento de información mediante entes llamados
                    archivos y directorios.

                \item
                    \textbf{Sistemas Paralelos: }

                    Sistemas utilizados para el multiprocesamiento compuesto por un conjunto
                    de procesadores que comparten el reljo, memoria y buses del equipo, por lo
                    que se conocen como fuertemente acoplados.

                \item
                    \textbf{Sistemas Distribuidos: }

                    istemas utilizados para el multiprocesamiento compuesto por un conjunto
                    de sistemas de cómputo completo que manejan de forma independiente cosas
                    como el reloj, la memoria o los buses. Por esto se le conoce como debilmente
                    acoplados.

            \end{itemize}



    % ===============================================================================
    % ===========          PARTES DEL SISTEMA OPERATIVO           ===================
    % ===============================================================================
    \chapter{Partes del Sistema Operativo}

        % ==============================================
        % =============    VISTA GENERAL    ============
        % ==============================================
        \clearpage
        \section{Vista General}

            Un Sistema Operativo como cualquier otro software sigue un modelo de ingeniería de
            software para su diseño y construcción, en modelo que se usa es el denominado por capas,
            este modelo nos dice que cada capa se encarga de realizar una funcionalidad concreta
            dentro del sistema operativo.

            Un sistema operativo normalmente está integrado por las siguientes capas:
            \begin{itemize}
                \item Hardware
                \item Kernel
                \item Servicios        
                \item Aplicaciones
            \end{itemize}  

            Estas capas para llevar a cabo sus funciones requieren comunicarse con cada adyacentes,
            para lograr esta comunicación se usan las interfaces,
            estas son:
            \begin{itemize}
                \item Interfaz de Comandos
                \item Interfaz de Llamafas al Sistema
                \item Interfaz de Interrupciones        
            \end{itemize}  


            Graficamente las podemos ver como:

            \begin{figure}[h!]
                \centering
                \includegraphics[width=0.55\textwidth]{Capas}
            \end{figure}

        % ==============================================
        % =============        CAPAS        ============
        % ==============================================
        \clearpage
        \section{Capas}    

            \begin{itemize}
                \item
                    \textbf{Capa de Aplicaciones}

                    Esta capa se encarga de mantener cualquier aplicación que el usuario ejecutará
                    en el sistema operativo, siendo la capa con la que el usuario tendrá contacto.

                    Se encarga de mantener todas las aplicaciones que tu conoces normalmente.


                \item
                    \textbf{Capa de Servicios}

                    Esta capa se encarga de mantener los servicio que apoyan el funcionamiento
                    del sistema operativo, teniendo servicios de seguridad, de mantenimiento,
                    entre otras.

                    A veces se suele unir y dividir sus funciones entre la capa de aplicaciones
                    y el kernel.


                \item
                    \textbf{Capa de Kernel o Núcleo}

                    Esta es la capa principal del sistema operativo, esta mantiene a los 5
                    administradores que componen a todo sistema operativo.


                \item
                    \textbf{Capa de Hardware}

                    Esta es la capa mas baja del sistema, se encarga de mantener todas las interfaces
                    de comunicación con el hardware.

            \end{itemize}


        % ==============================================
        % =========       INTERFACES        ============
        % ==============================================
        \clearpage
        \section{Interfaces}    
            
            \begin{itemize}
                \item
                    \textbf{Interfaz de Comandos}

                    Esta interfaz comunica a la capa de aplicaciones con la de
                    servicios o a la capa del kernel.

                    Esta compuesta de por todos los comandos disponibles en el sistema
                    operativo, siendo la interfaz de interacción inmediata que el usuario
                    posee para comunicarse con el sistema operativo.

                \item
                    \textbf{Interfaz de Llamadas al Sistema}

                    Esta interfaz comunica a la capa de servicios con la de
                    aplicaciones o a la capa del kernel.

                    Esta compuesta por unas APIs (es decir funciones o métodos)
                    que el sistema operativo pone a dispoción de los usuarios a
                    través de un lenguaje de alto nivel, por esto se le conoce como una
                    comunicación indirecta.

                \item
                    \textbf{Interfaz de Interrupciones}

                    Esta interfaz comunica a la capa del kernel con la del hardware.

                    Esta compuesta por un conjunto de interrupciones o servicios de 
                    interrupción que el sistema operativo pone a dispoción de los
                    usuarios por medio de un lenguaje de programación de bajo nivel,
                    por esto se le conoce como una comunicación indirecta.

            \end{itemize}


\end{document}










